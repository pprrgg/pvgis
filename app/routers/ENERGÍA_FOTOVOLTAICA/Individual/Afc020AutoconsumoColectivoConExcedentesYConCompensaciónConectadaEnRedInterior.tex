


\documentclass[a4paper,10pt,twocolumn]{article}
\usepackage[top = 2cm, bottom = 2cm, left = 2cm, right = 2cm, asymmetric]{geometry} % Especificar los márgenes según la norma
\usepackage[spanish]{babel}    

\usepackage{graphicx}  % Para incluir imágenes si es necesario
\usepackage{amsmath, amssymb}  % Para expresiones matemáticas
\usepackage{fancyhdr}  % Para personalizar encabezados
\usepackage{chngcntr}  % Para cambiar la numeración de apartados y subsecciones
\usepackage{tocbibind}  % Para incluir bibliografía en la tabla de contenidos
\usepackage{appendix}  % Para el formato de anexos
\usepackage{lipsum}  % Para generar texto de relleno (dummy text)
\usepackage{geometry}  % Para personalizar los márgenes
\usepackage{multicol}  % Para columnas
\usepackage{titlesec}  % Para personalizar los títulos de las secciones
\usepackage[utf8]{inputenc}    
\usepackage[T1]{fontenc}       
\usepackage[pdftex,pdfencoding=auto, colorlinks=true, linkcolor=black, citecolor=black, filecolor=black, urlcolor=black]{hyperref}
\usepackage{tocloft}           
\usepackage{booktabs}          % Tablas profesionales
\usepackage{float} % Paquete necesario para usar la opción [H]
\usepackage{array}
\usepackage{longtable}
\usepackage{circuitikz}
\usepackage{tikz}
\usepackage{tikz-cd}


\usepackage{qrcode}            % Paquete para generar QR
\usepackage{tabularx} % Agregar en el preámbulo
\usepackage[absolute,overlay]{textpos} % Para posicionar objetos libremente

\pagestyle{fancy}
\fancyhf{}
\fancyhead[L]{AUTOCONSUMO COLECTIVO CON EXCEDENTES Y CON COMPENSACIÓN CONECTADA EN RED INTERIOR}  % Nombre del documento en el encabezado izquierdo
\fancyhead[C]{}  % Centro vacío
\fancyhead[R]{\thepage}  % Numeración de páginas en el encabezado derecho
% \renewcommand{\thepage}{\arabic{page}}  % Asegura la numeración de páginas en números arábigos

% Definición de los anexos
\newcommand{\annex}[1]{\section*{Anexo #1} \addcontentsline{toc}{section}{Anexo #1}}
\addto\captionsspanish{%
  \renewcommand{\tablename}{Tabla}%
  \renewcommand{\listtablename}{Índice de tablas}%
}
\title{{\small Ref.:\uppercase{Afc020}}\\{\textbf{AUTOCONSUMO COLECTIVO CON EXCEDENTES Y CON COMPENSACIÓN CONECTADA EN RED INTERIOR}}}

\author{
    \begin{minipage}{0.35\textwidth}
    \centering
    Autor\\
    Juan Pérez García \\
    Ingeniero\\
    +34 600 123 456\\
    juan.perez@example.com\\
    \end{minipage}%
    \hfill
    \begin{minipage}{0.35\textwidth}
    \centering
    Promotor\\
    Empresa Ejemplo S.L. \\
    %\\
    +34 622 345 678\\
    info@empresaejemplo.com\\    
    \end{minipage}%
    \hfill
    \begin{minipage}{0.35\textwidth}
    \centering
    %\\
    % \\
    %\\
    %\\
    %\\    
    \end{minipage}%
}

\date{\today}
\newcommand{\MostrarVariablesAlFinal}{}


\begin{document}
\begin{Form}

	% Mostrar el título
	\maketitle
	% \onecolumn

	\begin{abstract}
		Este es un ejemplo de resumen para un artículo en formato general. Aquí se debe proporcionar una visión general del contenido del artículo.
	\end{abstract}

	% Iniciar el formato de dos columnas

	% Reemplazamos el entorno IEEEkeywords por una lista de palabras clave
	\noindent\textbf{Palabras clave:} Ejemplo, LaTeX, Formato General, Documentación, IEEE.

	\tableofcontents  % Tabla de contenidos
	\listoffigures    % Lista de figuras
	\listoftables     % Lista de tablas

	\newpage

	% http://localhost:8080/#N4Igdg9gJgpgziAXAbVABwnAlgFyxMJZABgBoBGAXVJADcBDAGwFcYkQAlAERAF9T0mXPkIpyFanSat2AYQDiABT4CQGbHgJEATKWKSGLNohDztKwRpE7S2g9OMhZ5-peFaxe+0fbzyFtSFNUWQAZi8aQxkTeUJXQKsPZHE7SIc5f3j1dxDw1KkfE1k4yRgoAHN4IlAAMwAnCABbJDIQHAgkcnj6ps6adqRQ7obmxF02jsQu1R7RgBZ+yenakaQAVkXO4d6pzcQANm3R8QmkAHZeSl4gA
	\begin{tikzcd}
		& GGGG \arrow[d]                         & G2 \arrow[ld] & Gn \arrow[lld] \\
		RD \arrow[r] & CGP \arrow[rd] \arrow[d] \arrow[rrd] &               &                \\
		& C1                                   & C2            & Cn
	\end{tikzcd}



	
            \begin{table}[h!]
                \resizebox{0.45\textwidth}{!}{%
                \begin{tabular}{p{2.5cm}|lll}
\toprule
Energía consumida & kWh & €/kWh & €/mes \\
\midrule
PUNTA & 13.50 & 0,074409 & 1.00 \\
LLANO & 10.50 & 0,028470 & 0.30 \\
VALLE & 108.00 & 0,003034 & 0.33 \\
Coste energía & 132.00 & 0,150000 & 19.80 \\
Compensación excedentes FV & - & - & -14.56 \\
TOTAL & - & - & 6.87 \\
\bottomrule
\end{tabular}

                }
                \caption{Energía consumida}
            \end{table}
            

	
            \begin{table}[h!]
                \resizebox{0.45\textwidth}{!}{%
                \begin{tabular}{p{2.5cm}|lll}
\toprule
Potencia contratada & kW & €/kW/año & €/mes \\
\midrule
PUNTA & 66.00 & 26,164043 & 10.75 \\
VALLE & 66.00 & 1,143132 & 0.47 \\
Margen comercialización fijo & 66.00 & 3,113000 & 1.28 \\
TOTAL & - & - & 12.50 \\
TOTAL & - & - & 25.00 \\
\bottomrule
\end{tabular}

                }
                \caption{Potencia}
            \end{table}
            

	
            \begin{table}[h!]
                \resizebox{0.45\textwidth}{!}{%
                \begin{tabular}{p{2.5cm}|ll}
\toprule
Concepto &   & €/mes \\
\midrule
Subtotal &             & 19.37 \\
Impuesto eléctrico & 0,5\% & 0.10 \\
Alquiler contador & 30 días & 0.81 \\
Subtotal &  & 20.28 \\
IVA  & 5\% & 1.01 \\
TOTAL FACTURA &  & 21.29 \\
TOTAL & - & 62.86 \\
\bottomrule
\end{tabular}

                }
                \caption{Total factura}
            \end{table}
            

    \begin{figure}[H]
                
                \includegraphics[width=.5\textwidth]{/tmp/tmphzkdmo5y.png}
                \caption{Flujo de Caja y Acumulado}
                
                \label{fig:dfassssdfsa}
                \end{figure}
                

	% Capítulo 1
	\section{Introducción}
	Este es un ejemplo de documento con formato general. En esta sección se explica el objetivo general del documento.

	\subsection{Objetivo}
	El objetivo de este artículo es mostrar cómo adaptar el formato general para cumplir con normas específicas como la UNE 50135:1996.

	\subsubsection{Descripción del proyecto}
	\lipsum[1] % Texto de relleno generado con lipsum

	% Capítulo 2
	\section{Metodología}
	Aquí se describe la metodología utilizada para alcanzar los objetivos planteados.



	\subsubsection{Método de análisis}
	Descripción del método utilizado para el análisis de los datos.


	\section{Mecanismo de Compensación Simplificada}
	El \textbf{Artículo 14} del \textbf{Real Decreto 244/2019} establece el \textbf{Mecanismo de Compensación Simplificada}, que permite a los consumidores con instalaciones de autoconsumo compensar los excedentes de energía eléctrica vertidos a la red con el consumo de energía de la red en períodos posteriores. Este mecanismo es aplicable para instalaciones de hasta 100 kW.

	\subsection{¿Cómo funciona?}
	El mecanismo permite que la energía excedente inyectada a la red se compense con la energía consumida de la red en la factura eléctrica. La compensación se calcula de la siguiente manera:

	\begin{equation}
		\text{Compensación} = E_{\text{excedente}} \times P_{\text{compensación}}
	\end{equation}

	Donde:
	\begin{itemize}
		\item \( E_{\text{excedente}} \): Energía excedente vertida a la red (en kWh).
		\item \( P_{\text{compensación}} \): Precio de compensación acordado con la comercializadora (en €/kWh).
	\end{itemize}

	La compensación se descuenta del coste total de la energía consumida de la red:

	\begin{equation}
		\text{Coste total} = E_{\text{consumida}} \times P_{\text{consumo}} - \text{Compensación}
	\end{equation}

	\subsection{Ejemplo práctico}
	Supongamos los siguientes datos para un período de facturación:
	\begin{itemize}
		\item Energía consumida de la red: \( E_{\text{consumida}} = 500 \, \text{kWh} \).
		\item Energía excedente vertida a la red: \( E_{\text{excedente}} = 300 \, \text{kWh} \).
		\item Precio de la energía consumida: \( P_{\text{consumo}} = 0.15 \, \text{€/kWh} \).
		\item Precio de compensación: \( P_{\text{compensación}} = 0.05 \, \text{€/kWh} \).
	\end{itemize}

	Aplicando las fórmulas anteriores:
	\begin{enumerate}
		\item Calculamos la compensación por excedentes:
		      \[
			      \text{Compensación} = 300 \, \text{kWh} \times 0.05 \, \text{€/kWh} = 15 \, \text{€}
		      \]

		\item Calculamos el coste de la energía consumida:
		      \[
			      \text{Coste consumida} = 500 \, \text{kWh} \times 0.15 \, \text{€/kWh} = 75 \, \text{€}
		      \]

		\item Aplicamos la compensación:
		      \[
			      \text{Total a pagar} = 75 \, \text{€} - 15 \, \text{€} = 60 \, \text{€}
		      \]
	\end{enumerate}

	\subsection{Consideraciones importantes}
	\begin{itemize}
		\item La compensación no puede superar el coste de la energía consumida de la red. Es decir, no se puede obtener un saldo positivo a favor del consumidor.
		\item Este mecanismo solo es aplicable para instalaciones de autoconsumo con una potencia instalada menor o igual a 100 kW.
		\item El precio de compensación (\( P_{\text{compensación}} \)) lo fija la comercializadora, pero no puede superar el valor de mercado de la energía.
	\end{itemize}

	% Capítulo 4
	\section{Conclusión}
	Conclusión del artículo con un resumen de los hallazgos más importantes.

	\lipsum[4]  % Texto de relleno

	% Anexos
	\annex{A}
	\subsection{Cálculos adicionales}
	Detalles de los cálculos adicionales realizados en el informe.

	\lipsum[5]  % Texto de relleno para anexos

	\annex{B}
	\subsection{Datos experimentales}
	Descripción de los datos experimentales adicionales.



	% Bibliografía
	\newpage
	\begin{thebibliography}{9}
		\bibitem{une50135}
		\href
		{
			https://www.idae.es/tecnologias/energias-renovables/oficina-de-autoconsumo/guias-tecnicas-sobre-autoconsumo
		}
		{
			Guías técnicas y Formación sobre autoconsumo
		}
		, IDAE
		\bibitem{alarma}
		\href
		{
			https://www.idae.es/publicaciones/guia-profesional-de-tramitacion-del-autoconsumo
		}
		{
			Guía IDAE 021: Guía Profesional de Tramitación del Autoconsumo (edición v.6), D Ejemplos, p159
		}
		, IDAE

		\bibitem{normativa}
		Guía IDAE 026: \href{https://www.idae.es/publicaciones/guia-de-autoconsumo-colectivo}{Guía IDAE 026: Guía de autoconsumo colectivo (versión v.2.1)}. IDAE.
	\end{thebibliography}







	% P A R A M E T R O S 
	%---------------------

	\TextField[name=I0 ,width=0cm]{}
	\TextField[name=Bt,width=0cm]{}


	%==========================================




\end{Form}

\end{document}


