\documentclass[10pt, a4paper, twoside, twocolumn]{article}
\usepackage[top = 2cm, bottom = 2cm, left = 2cm, right = 2cm, asymmetric]{geometry} % Especificar los márgenes según la norma

% Paquetes básicos
\usepackage[spanish]{babel}    
\usepackage[utf8]{inputenc}    
\usepackage[T1]{fontenc}       
\usepackage{geometry}          
% \geometry{left=3cm, right=3cm, top=3cm, bottom=3cm}
\usepackage{graphicx}          
\usepackage[pdftex,pdfencoding=auto, colorlinks=true, linkcolor=black, citecolor=black, filecolor=black, urlcolor=black]{hyperref}
\usepackage{fancyhdr}          
\usepackage{titlesec}          
\usepackage{tocloft}           
\usepackage{booktabs}          % Tablas profesionales
\usepackage{float} % Paquete necesario para usar la opción [H]
\usepackage{array}
\usepackage{booktabs}
\usepackage{lipsum} % Para generar texto de ejemplo
\usepackage{amsmath}

% 

\usepackage[spanish]{babel} % Para utilizar el idioma español y sus convenciones
\usepackage{qrcode}            % Paquete para generar QR
\let\OldTextField\TextField
\renewcommand{\TextField}[2][]{%
  \raisebox{-0.1ex}{\OldTextField[height=.95em,  bordercolor={1 1 1}, backgroundcolor={1 1 1},#1]{#2}}%
}
% Encabezado y pie de página
\pagestyle{fancy}
\fancyhead[L]{\textbf{Conversión de Coche de Gasolina a Eléctrico}}
\fancyhead[R]{\thepage}
\fancyfoot[C]{}

% Configuración de títulos
\titleformat{\section}{\large\bfseries}{\thesection.}{1em}{}
\titleformat{\subsection}{\normalsize\bfseries}{\thesubsection.}{1em}{}

\begin{document}
\begin{Form}

	% Portada
	\begin{titlepage}
		\centering
		{\Large \textbf{INFORME TECNICO}} \\
		\vspace{3cm}
		{\huge \textbf{Conversión de Coche de Gasolina a Eléctrico}} \\
		\vspace{3cm}
		\fbox{{\Huge{TEC}}\textcolor{white}{\Huge\colorbox{black}{I}}}{\tiny I\hspace{0.30cm}N\hspace{0.30cm}G\hspace{0.30cm}E\hspace{0.30cm}N\hspace{0.30cm}I\hspace{0.30cm}E\hspace{0.30cm}R\hspace{0.30cm}I\hspace{0.30cm}A} \\
		\vspace{4cm}
		\begin{center}
			\begin{tabular*}{0.5\textwidth}{@{\extracolsep{\fill}}  p{0.3\textwidth} p{0.3\textwidth} @{}}
				\textbf{Referencia:} 			&  IAF010  \\
				% \textbf{Ubicación:} 			&  \TextField[name=Ubicación,width=5cm]{}  \\
				\textbf{Ahorro [kWh/a]:} 		& pp.ahorroanualdeenergiakwh  \\
				\textbf{Autor:} 				&  DocTec  \\
				\textbf{Entidad:}				&  TEC Ingeniería  \\
			\end{tabular*}
		\end{center}
		\vfill
		{\large 	\today  } \\

		% QR en la esquina inferior derecha
		\begin{flushright}
			\qrcode{https://doctec.blog//} \\
			https://doctec.blog//
		\end{flushright}
	\end{titlepage}

	% Índice
	\tableofcontents

	\noindent\hrulefill\ + \hrulefill


	\section{Introducción}
	Este informe analiza la viabilidad técnica y económica de convertir un coche de gasolina a un vehículo eléctrico, considerando los costos de conversión y el ahorro en combustible a lo largo del tiempo.

	\section{Consumo y Costes de Combustible}
	Se considera un coche de gasolina con un consumo medio de 6.5 L/100 km y un kilometraje anual de 15,000 km. Con un precio promedio de 1.75 €/L, el gasto anual en gasolina es:
	\begin{equation}
		C_{gasolina} = 6.5 \times \frac{15,000}{100} \times 1.75 \approx 1,706 \,\text{€}
	\end{equation}

	Para un coche eléctrico con un consumo de 17 kWh/100 km y un costo de electricidad de 0.20 €/kWh, el gasto anual en electricidad es:
	\begin{equation}
		C_{eléctrico} = 17 \times \frac{15,000}{100} \times 0.20 \approx 510 \,\text{€}
	\end{equation}

	\section{Coste de Conversión}
	El costo de conversión incluye:
	\begin{itemize}
		\item Batería (40 kWh): 6,000 €
		\item Motor eléctrico y controlador: 3,000 €
		\item Sistema de carga y cableado: 1,500 €
		\item Mano de obra e instalación: 2,500 €
	\end{itemize}
	El costo total estimado de conversión es:
	\begin{equation}
		C_{conversión} = 6,000 + 3,000 + 1,500 + 2,500 = 13,000 \,\text{€}
	\end{equation}

	\section{Ahorro y Retorno de la Inversión}
	El ahorro anual en combustible es:
	\begin{equation}
		A = 1,706 - 510 = 1,196 \,\text{€}
	\end{equation}

	El período de amortización de la conversión es:
	\begin{equation}
		T = \frac{13,000}{1,196} \approx 10.9 \text{ años}
	\end{equation}

	\section{Conclusión}
	La conversión de un coche de gasolina a eléctrico representa una inversión inicial significativa de 13,000 €, con un ahorro anual de 1,196 €. El período de amortización es de aproximadamente 11 años, tras los cuales el usuario comenzará a obtener beneficios económicos adicionales, además de reducir su impacto ambiental.


	\section{Referencias}
	\begin{itemize}
		\item \href{
			      https://icaen.gencat.cat/es/detalls/publicacio/Num.-9-Installacio-dinfraestructura-de-recarrega-del-vehicle-electric

		      }
		      {
			      Instalación de infraestructura de recarga del vehículo eléctrico
		      }
		      , ICAEN,
		\item \href{
			https://eurospor100km.energia.gob.es/Paginas/Index.aspx

		      }
		      {
				Información comparativa sobre el coste de los combustibles de automoción en €/100km
		      }
		      , ICAEN,


	\end{itemize}





	% \graphicspath{{./docs/INFORMES_TECNICOS/EFICIENCIA_ENERGETICA/}} % Define el directorio base
	% \includepdf[pages=-]{frv010/cc.pdf} % Importa todas las páginas





	% P A R A M E T R O S 
	%---------------------

	\TextField[name=I0 ,width=0cm]{}
	\TextField[name=Bt,width=0cm]{}


	%==========================================




\end{Form}

\end{document}













\begin{lstlisting}
	
#######################################################################################################
#######################################################################################################




#######################################################################################################
######################################################################################################
#######################################################################################################
#######################################################################################################

\end{lstlisting}
